\documentclass[12pt,letterpaper]{article}
\usepackage[latin1]{inputenc}
\usepackage[spanish]{babel}
\usepackage{amsmath}
\usepackage{amsfonts}
\usepackage{amssymb}
\usepackage{graphicx}
\usepackage{enumerate}
\usepackage[left=4cm,right=3cm,top=4cm,bottom=4cm]{geometry}
\author{Emmanuel Isaac Pezo Ramirez }
\title{Pr�ctica 01}
\begin{document}
\maketitle
\begin{enumerate}
\item  Demuestre que $\mathcal{F}$ es una $\sigma$ - �lgebra de subconjuntos de si, y solo si,satisface las siguientes propiedades:
\begin{enumerate}[a) ]

\item $\phi \in \mathcal{F}$\\[0.2cm]
Demostraci�n\\[0.2cm]
$\Omega \in \mathcal{F}$ y $ \mathcal{F}$ colecci�n cerrada\\[0.2cm]
$\Omega^c = \phi\in\mathcal{F}$\\[0.2cm]

\item  $A\in \mathcal{F}\longrightarrow A^{c} \in\mathcal{F}$\\[0.2cm]
Demostraci�n\\[0.2cm]
Supongamos que $\mathcal{F}$ es una $\sigma$ - �lgebra sobre $\Omega$\\[0.2cm]
$\Rightarrow \Omega \in\mathcal{F}$ como $\phi \in \mathcal{F}$ a su vez verifica que $\phi= {\Omega^c} \in  \mathcal{F}$

\item  $A_1,A_2,... \in \mathcal{F} \longrightarrow\displaystyle\bigcup_{n=1}^{\infty}{A_n \in \mathcal{F}}$\\[0.2cm]
Demostraci�n\\[0.2cm]
Sean 


\end{enumerate}

\item Sea $\mathcal{F}\,una \,\sigma �lgebra; demuestre que\,\mathcal{F}^c es una\,\sigma - �lgebra definida por: \mathcal{F}^c= \lbraceA^c: A\in\mathcal{F}\rbrace$

Soluci�n\\[0.2cm]
$\mathcal{F}\,es una \sigma\,�lgebra entonces se cumple que\, A\in \mathcal{F}\longrightarrow A^{c} \in\mathcal{F}$

\item Sea $\{A_n\}_{n\in\mathbb{N}}$ la sucesi�n de eventos, definida por:
\[A_n = \begin{cases} 
      A  & \mbox{si } n= 1,3,5,...        \\[0.5cm]
      A^c  & \mbox{si } n = 2,4,6,...
 \end{cases} \]
Determine el: $\lim\limits_{x \to \infty}A_n$\\[0.2cm]
Soluci�n\\[0.2cm]
$\lim\limits_{n \to \infty}{inf}A_n =\displaystyle\bigcup_{n=1}^{\infty}\displaystyle\bigcap_{k=n}^{\infty}A_k = \displaystyle\bigcup_{n=1}^{\infty}\left(A \cap A^c\right)= \displaystyle\bigcup_{n=1}^{\infty}\phi = \phi $\\[0.2cm]
$\lim\limits_{n \to \infty}{sup}A_n= \displaystyle\bigcap_{n=1}^{\infty}\displaystyle\bigcup_{k=n}^{\infty}A_k = \displaystyle\bigcap_{n=1}^{\infty}\left(A \cap A^c\right)= \displaystyle\bigcap_{n=1}^{\infty}\Omega = \Omega $\\[0.2cm]
$\Rightarrow \lim\limits_{n \to \infty}{inf}A_n \not = \lim\limits_{n \to \infty}{sup}A_n $\\[0.2cm]
$\therefore \lim\limits_{n \to \infty}A_n \nexists$

\item Sea $\{A_n\}_{n\in\mathbb{N}}$ la sucesi�n de eventos, definida por:
\[A_n = \begin{cases} 
  A  & \mbox{si } \left[-1/n,0\right]  \\[0.5cm]
  A^c  & \mbox{si } \left[0,1/n\right] \end{cases} \]
Determine el: $\lim\limits_{x \to \infty}A_n$

Soluci�n\\[0.2cm]
$\lim\limits_{n \to {infty}{inf}A_n=\displaystyle\bigcup_{n=1}^{\infty}\displaystyle\bigcap_{k=n}^{\infty}A_k=$

\item Sean $A_1,A_2,...$ eventos aleatorios, demuestre :
\begin{enumerate}[a) ]

\item $P\left(\displaystyle\bigcap_{i=1}^{n}{A_i}\right)\geq 1-\displaystyle\sum_{i=1}^{n}$ $P\left(A_{i}\right)^c$\\[0.2cm]
Demostraci�n\\[0.2cm]
Si: $P\left(\displaystyle\bigcap_{i=1}^{n}{A_i}\right)= P\left(\displaystyle\bigcup_{i=1}^{n}{A_i^c}\right)^c$\\[0.2cm]
$P\left(\displaystyle\bigcap_{i=1}^{n}{A_i}\right)= 1- P\left(\displaystyle\bigcup_{i=1}^{n}{A_i^c}\right)$\\[0.2cm]
Recuerde: $P\left(\displaystyle\bigcup_{i=1}^{n}{A_i}\right) \leqslant\displaystyle\sum_{i=1}^{n}$ $P\left(A_{i}\right)$\\[0.2cm]
$\Rightarrow P\left(\displaystyle\bigcap_{i=1}^{n}{A_i}\right)= 1- P\left(\displaystyle\bigcup_{i=1}^{n}{A_i^c}\right) \geqslant 1-\displaystyle\sum_{i=1}^{n}$ $P\left(A_{i}\right)^c $

\item Si $P\left(A_{i}\right) \geq 1-e$ para i=1,2,...,n entonces $P\left(\displaystyle\bigcap_{i=1}^{n}\right)\geq 1-ne$\\[0.2cm]

Demostraci�n\\[0.2cm]
tenemos: $P\left(A_{i}\right)\geqslant 1-e$\\[0.2cm]
	$\Rightarrow e \geqslant 1-P\left(A_{i}\right)$\\[0.2cm]
	$ \displaystyle\prod_{i=1}^{n}e \geqslant \displaystyle\prod_{i=1}^{n}P\left(A_{i}\right)^c$\\[0.2cm]
$ne \geqslant \displaystyle\prod_{i=1}^{n}\left(1- P{\left(A_{i}\right)}\right)\Rightarrow ne\geqslant 1-P\left(\displaystyle\bigcap_{i=1}^{n}A_i\right)$ \\[0.2cm]
$\therefore P\left(\displaystyle\bigcap_{i=1}^{n}A_i\right)\geqslant 1-ne $
			 
		
\item $P\left(\displaystyle\bigcap_{k=1}^{\infty}{A_k}\right) \geq 1-\displaystyle\sum_{i=1}^{\infty}P\left(A_{k}^{c}\right)$

\end{enumerate}

\item Demuestre las desigualdades de Boole\\[0.2cm]
\begin{enumerate}[a) ]

\item $P (\left\displaystyle\bigcup_{n=1}^{\infty}A_{n})\leq\sum_{n=1}^{\infty}P\ (A_{n})$\\[0.2cm]

Demostraci�n \\[0.2cm]
Sean $\left A_{1}=B_{1}}\\[0.2cm]$
$\left\ B_{n}=A_{n}-\displaystyle\bigcup_{k=1}^{n-1} A_k\right; n=1,2,3,.... \\[0.2cm]$

$\left\displaystyle\bigcup_{n=1}^{\infty}A_n=\displaystyle\bigcup_{n=1}^{\infty}B_n \\[0.2cm]$

$\left\ B_{n} \cap B_{m}; si: n \neq m\\[0.2cm]$

$\left\ B_{n} \subseteq A{n}\\[0.2cm]$ 

Por lo tanto:\\[0.2cm]
$P (\left\displaystyle\bigcup_{n=1}^{\infty}A_{n})=P (\displaystyle\bigcup_{n=1}^{\infty}B_{n})$

$P (\left\displaystyle\bigcup_{n=1}^{\infty}A_{n})=\sum_{n=1}^{\infty}P\ (B_{n})$

$P (\left\displaystyle\bigcup_{n=1}^{\infty}A_{n})\leq\sum_{n=1}^{\infty}P\ (A_{n})$\\[0.2cm]

\item $P (\left\displaystyle\bigcup_{n=1}^{\infty}A_{n})\geq 1-\sum_{n=1}^{\infty}P\ (A^c_{n})$\\[0.2cm]


Demostraci�n: \\[0.2cm]

Por las leyes de De Morgan tenemos:\\[0.2cm]

$P (\left\displaystyle\bigcup_{n=1}^{\infty}A_{n})=1-(\left\displaystyle\bigcup_{n=1}^{\infty}A^c_{n})$

De la demostraci�n anterior tenemos que:

$P (\left\displaystyle\bigcup_{n=1}^{\infty}A_{n})\leq\sum_{n=1}^{\infty}P\ (A_{n})$\\[0.2cm]

Luego:

$P (\left\displaystyle\bigcup_{n=1}^{\infty}A_{n})\geq 1-\sum_{n=1}^{\infty}P\ (A^c_{n})$\\[0.2cm]

\end{enumerate}

\item Sea $\{A_n\}n \in {N}$ una sucesi�n de eventos. Demuestre que:
\begin{enumerate}[a) ]

\item $\left(\displaystyle\lim_{n\longrightarrow \infty}{inf}A_n\right)^c =\displaystyle\lim_{n\longrightarrow \infty} {sup}{A_n^c} $

Demostraci�n:\\[0.2cm]
$\left(\displaystyle\lim_{n\longrightarrow \infty}{inf}A_k\right)^c=(\displaystyle\bigcup_{n=1}^{\infty}\displaystyle\bigcap_{k=n}^{\infty}A_{k})^c$ 

$(\displaystyle\bigcup_{n=1}^{\infty}\displaystyle\bigcap_{n=1}^{\infty}A_{k})^c=\displaystyle\bigcap_{n=1}^{\infty}\displaystyle\bigcup_{n=1}^{\infty}A_{k}^c\\[0.2cm]$

Pero:\\[0.2cm]
$\displaystyle\lim_{n\longrightarrow \infty} {sup}{A_n^c}=\displaystyle\bigcap_{n=1}^{\infty}\displaystyle\bigcup_{k=n}^{\infty}A_{k}^c$ 
 
$\therefore\left(\displaystyle\lim_{n\longrightarrow \infty}{inf}A_n\right)^c =\displaystyle\lim_{n\longrightarrow \infty} {sup}{A_n^c}\\[0.2cm]$

\item $\left(\displaystyle\lim_{n\longrightarrow \infty}{inf}A_n\right)^c =\displaystyle\lim_{n\longrightarrow \infty} {inf}{A_n^c} $

Demostraci�n:\\[0.2cm]
 $\left(\displaystyle\lim_{n\longrightarrow \infty}{inf}A_n\right)^c=(\displaystyle\bigcap_{n=1}^{\infty}\displaystyle\bigcup_{k=n}^{\infty}A_{k})^c$
 
 $(\displaystyle\bigcap_{n=1}^{\infty}\displaystyle\bigcup_{k=n}^{\infty}A_{k})^c=\displaystyle\bigcup_{n=1}^{\infty}\displaystyle\bigcap_{k=n}^{\infty}A_{k}^c$
 
 $\displaystyle\bigcup_{n=1}^{\infty}\displaystyle\bigcap_{k=n}^{\infty}A_{k}^c=\displaystyle\lim_{n\longrightarrow \infty} {inf}{A_n^c} $
 
 Entonces:\\[0.2cm]
 $\left(\displaystyle\lim_{n\longrightarrow \infty}{inf}A_n\right)^c =\displaystyle\lim_{n\longrightarrow \infty} {inf}{A_n^c} $
 

\item $P\left(\displaystyle\lim_{n\longrightarrow \infty}{inf}A_n\right) =1-P\left(\displaystyle\lim_{n\longrightarrow \infty}{sup}A_n^c\right)$ 

Demostraci�n:\\[0.2cm]
$P\left(\displaystyle\lim_{n\longrightarrow \infty}{inf}A_n\right)=[P (\displaystyle\bigcup_{n=1}^{\infty}\displaystyle\bigcap_{k=n}^{\infty}A_{k})^c]^c$

$[P (\displaystyle\bigcup_{n=1}^{\infty}\displaystyle\bigcap_{k=n}^{\infty}A_{k})^c]^c=1-P(\displaystyle\bigcup_{n=1}^{k=n}\dislaystyle\bigcap_{n=1}^{k=n}A_{k})^c$

$1-P(\displaystyle\bigcup_{n=1}^{k=n}\dislaystyle\bigcap_{n=1}^{k=n}A_{k})^c=1-P(\displaystyle\bigcap_{n=1}^{k=n}\displaystyle\bigcup_{k=n}A_{k}^c)$

$1-P(\displaystyle\bigcap_{n=1}^{k=n}\displaystyle\bigcup_{k=n}A_{k}^c)=1-P\left(\displaystyle\lim_{n\longrightarrow \infty}{sup}A_n^c\right)$ 

$\therefore P\left(\displaystyle\lim_{n\longrightarrow \infty}{inf}A_n\right) =1-P\left(\displaystyle\lim_{n\longrightarrow \infty}{sup}A_n^c\right)$


\end{enumerate}

\item Encuentre las condiciones sobre los eventos $A_1$ y $A_2$ para que la siguiente sucesi�n sea convergente.
\[A_n = \begin{cases} 
     A_1  & \mbox{si } $n es impar$   \\[0.5cm]
      A_2  & \mbox{si }$ n es par$
 \end{cases} \]

Soluci�n: 

\end{document}

